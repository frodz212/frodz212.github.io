\documentclass{article}\usepackage[]{graphicx}\usepackage[]{xcolor}
% maxwidth is the original width if it is less than linewidth
% otherwise use linewidth (to make sure the graphics do not exceed the margin)
\makeatletter
\def\maxwidth{ %
  \ifdim\Gin@nat@width>\linewidth
    \linewidth
  \else
    \Gin@nat@width
  \fi
}
\makeatother

\definecolor{fgcolor}{rgb}{0.345, 0.345, 0.345}
\newcommand{\hlnum}[1]{\textcolor[rgb]{0.686,0.059,0.569}{#1}}%
\newcommand{\hlstr}[1]{\textcolor[rgb]{0.192,0.494,0.8}{#1}}%
\newcommand{\hlcom}[1]{\textcolor[rgb]{0.678,0.584,0.686}{\textit{#1}}}%
\newcommand{\hlopt}[1]{\textcolor[rgb]{0,0,0}{#1}}%
\newcommand{\hlstd}[1]{\textcolor[rgb]{0.345,0.345,0.345}{#1}}%
\newcommand{\hlkwa}[1]{\textcolor[rgb]{0.161,0.373,0.58}{\textbf{#1}}}%
\newcommand{\hlkwb}[1]{\textcolor[rgb]{0.69,0.353,0.396}{#1}}%
\newcommand{\hlkwc}[1]{\textcolor[rgb]{0.333,0.667,0.333}{#1}}%
\newcommand{\hlkwd}[1]{\textcolor[rgb]{0.737,0.353,0.396}{\textbf{#1}}}%
\let\hlipl\hlkwb

\usepackage{framed}
\makeatletter
\newenvironment{kframe}{%
 \def\at@end@of@kframe{}%
 \ifinner\ifhmode%
  \def\at@end@of@kframe{\end{minipage}}%
  \begin{minipage}{\columnwidth}%
 \fi\fi%
 \def\FrameCommand##1{\hskip\@totalleftmargin \hskip-\fboxsep
 \colorbox{shadecolor}{##1}\hskip-\fboxsep
     % There is no \\@totalrightmargin, so:
     \hskip-\linewidth \hskip-\@totalleftmargin \hskip\columnwidth}%
 \MakeFramed {\advance\hsize-\width
   \@totalleftmargin\z@ \linewidth\hsize
   \@setminipage}}%
 {\par\unskip\endMakeFramed%
 \at@end@of@kframe}
\makeatother

\definecolor{shadecolor}{rgb}{.97, .97, .97}
\definecolor{messagecolor}{rgb}{0, 0, 0}
\definecolor{warningcolor}{rgb}{1, 0, 1}
\definecolor{errorcolor}{rgb}{1, 0, 0}
\newenvironment{knitrout}{}{} % an empty environment to be redefined in TeX

\usepackage{alltt}
\usepackage[sc]{mathpazo}
\renewcommand{\sfdefault}{lmss}
\renewcommand{\ttdefault}{lmtt}
\usepackage[T1]{fontenc}
\usepackage{geometry}
\geometry{verbose,tmargin=2.5cm,bmargin=2.5cm,lmargin=2.5cm,rmargin=2.5cm}
\setcounter{secnumdepth}{2}
\setcounter{tocdepth}{2}
\usepackage[unicode=true,pdfusetitle,
 bookmarks=true,bookmarksnumbered=true,bookmarksopen=true,bookmarksopenlevel=2,
 breaklinks=false,pdfborder={0 0 1},backref=false,colorlinks=false]
 {hyperref}
\hypersetup{
 pdfstartview={XYZ null null 1}}

\makeatletter
%%%%%%%%%%%%%%%%%%%%%%%%%%%%%% User specified LaTeX commands.
\renewcommand{\textfraction}{0.05}
\renewcommand{\topfraction}{0.8}
\renewcommand{\bottomfraction}{0.8}
\renewcommand{\floatpagefraction}{0.75}

\makeatother
\IfFileExists{upquote.sty}{\usepackage{upquote}}{}
\begin{document}



\title{\title{\title{\title{\title{\title{\title{}}}}}}}



\maketitle
The results below are generated from an R script.

\begin{knitrout}
\definecolor{shadecolor}{rgb}{0.969, 0.969, 0.969}\color{fgcolor}\begin{kframe}
\begin{alltt}
\hlcom{# Assignment: ASSIGNMENT 1}
\hlcom{# Name: Rodriguez, Felipe}
\hlcom{# Date: 2022-12-11}

\hlcom{## Create a numeric vector with the values of 3, 2, 1 using the `c()` function}
\hlcom{## Assign the value to a variable named `num_vector`}
\hlcom{## Print the vector}
\hlstd{num_vector} \hlkwb{<-} \hlkwd{c}\hlstd{(}\hlnum{3}\hlstd{,} \hlnum{2}\hlstd{,} \hlnum{1}\hlstd{)}

\hlcom{## Create a character vector with the values of "three", "two", "one" "using the `c()` function}
\hlcom{## Assign the value to a variable named `char_vector`}
\hlcom{## Print the vector}
\hlstd{char_vector} \hlkwb{<-} \hlkwd{c}\hlstd{(}\hlstr{"three"}\hlstd{,} \hlstr{"two"}\hlstd{,} \hlstr{"one"}\hlstd{)}
\hlstd{char_vector}
\end{alltt}
\begin{verbatim}
## [1] "three" "two"   "one"
\end{verbatim}
\begin{alltt}
\hlcom{## Create a vector called `week1_sleep` representing how many hours slept each night of the week}
\hlcom{## Use the values 6.1, 8.8, 7.7, 6.4, 6.2, 6.9, 6.6}
\hlstd{week1_sleep} \hlkwb{<-} \hlkwd{c}\hlstd{(}\hlnum{6.1}\hlstd{,} \hlnum{8.8}\hlstd{,} \hlnum{7.7}\hlstd{,} \hlnum{6.4}\hlstd{,} \hlnum{6.2}\hlstd{,} \hlnum{6.9}\hlstd{,} \hlnum{6.6}\hlstd{)}

\hlcom{## Display the amount of sleep on Tuesday of week 1 by selecting the variable index}
\hlstd{week1_sleep[}\hlnum{2}\hlstd{]}
\end{alltt}
\begin{verbatim}
## [1] 8.8
\end{verbatim}
\begin{alltt}
\hlcom{## Create a vector called `week1_sleep_weekdays`}
\hlcom{## Assign the weekday values using indice slicing}
\hlstd{week1_sleep_weekdays} \hlkwb{<-} \hlstd{week1_sleep[}\hlnum{1}\hlopt{:}\hlnum{7}\hlstd{]}

\hlcom{## Add the total hours slept in week one using the `sum` function}
\hlcom{## Assign the value to variable `total_sleep_week1`}
\hlstd{total_sleep_week1} \hlkwb{<-} \hlkwd{sum}\hlstd{(week1_sleep)}

\hlcom{## Create a vector called `week2_sleep` representing how many hours slept each night of the week}
\hlcom{## Use the values 7.1, 7.4, 7.9, 6.5, 8.1, 8.2, 8.9}
\hlstd{week2_sleep} \hlkwb{<-} \hlkwd{c}\hlstd{(}\hlnum{7.1}\hlstd{,} \hlnum{7.2}\hlstd{,} \hlnum{7.9}\hlstd{,} \hlnum{6.5}\hlstd{,} \hlnum{8.1}\hlstd{,} \hlnum{8.2}\hlstd{,} \hlnum{8.9}\hlstd{)}

\hlcom{## Add the total hours slept in week two using the sum function}
\hlcom{## Assign the value to variable `total_sleep_week2`}
\hlstd{total_sleep_week2} \hlkwb{<-} \hlkwd{sum}\hlstd{(week2_sleep)}

\hlcom{## Determine if the total sleep in week 1 is less than week 2 by using the < operator}
\hlstd{total_sleep_week1} \hlopt{<} \hlstd{total_sleep_week2}
\end{alltt}
\begin{verbatim}
## [1] TRUE
\end{verbatim}
\begin{alltt}
\hlcom{## Calculate the mean hours slept in week 1 using the `mean()` function}
\hlkwd{mean}\hlstd{(week1_sleep)}
\end{alltt}
\begin{verbatim}
## [1] 6.957143
\end{verbatim}
\begin{alltt}
\hlcom{## Create a vector called `days` containing the days of the week.}
\hlcom{## Start with Sunday and end with Saturday}
\hlstd{days} \hlkwb{<-} \hlkwd{c}\hlstd{(}\hlstr{"Sunday"}\hlstd{,} \hlstr{"Monday"}\hlstd{,} \hlstr{"Tuesday"}\hlstd{,} \hlstr{"Wednesday"}\hlstd{,} \hlstr{"Thursday"}\hlstd{,} \hlstr{"Friday"}\hlstd{,} \hlstr{"Saturday"}\hlstd{)}

\hlcom{## Assign the names of each day to `week1_sleep` and `week2_sleep` using the `names` function and `days` vector}
\hlkwd{names}\hlstd{(week1_sleep)} \hlkwb{<-} \hlkwd{c}\hlstd{(days)}
\hlkwd{names}\hlstd{(week2_sleep)} \hlkwb{<-} \hlkwd{c}\hlstd{(days)}

\hlcom{## Display the amount of sleep on Tuesday of week 1 by selecting the variable name}
\hlstd{week1_sleep[}\hlstr{"Tuesday"}\hlstd{]}
\end{alltt}
\begin{verbatim}
## Tuesday 
##     7.7
\end{verbatim}
\begin{alltt}
\hlcom{## Create vector called weekdays from the days vector}
\hlstd{weekdays} \hlkwb{<-} \hlstd{days[}\hlnum{2}\hlopt{:}\hlnum{6}\hlstd{]}

\hlcom{## Create vector called weekends containing Sunday and Saturday}
\hlstd{weekends} \hlkwb{<-} \hlkwd{c}\hlstd{(}\hlstr{"Sunday"}\hlstd{,} \hlstr{"Saturday"}\hlstd{)}

\hlcom{## Calculate the mean about sleep on weekdays for each week}
\hlcom{## Assign the values to weekdays1_mean and weekdays2_mean}
\hlstd{weekdays1_mean} \hlkwb{<-} \hlkwd{mean}\hlstd{(week1_sleep[weekdays])}
\hlstd{weekdays2_mean} \hlkwb{<-} \hlkwd{mean}\hlstd{(week2_sleep[weekdays])}

\hlcom{## Using the weekdays1_mean and weekdays2_mean variables,}
\hlcom{## see if weekdays1_mean is greater than weekdays2_mean using the `>` operator}
\hlstd{weekdays1_mean} \hlopt{>} \hlstd{weekdays2_mean}
\end{alltt}
\begin{verbatim}
## [1] FALSE
\end{verbatim}
\begin{alltt}
\hlcom{## Determine how many days in week 1 had over 8 hours of sleep using the `>` operator}
\hlstd{week1_sleep[days]}
\end{alltt}
\begin{verbatim}
##    Sunday    Monday   Tuesday Wednesday  Thursday    Friday  Saturday 
##       6.1       8.8       7.7       6.4       6.2       6.9       6.6
\end{verbatim}
\begin{alltt}
\hlcom{## Create a matrix from the following three vectors}
\hlstd{student01} \hlkwb{<-} \hlkwd{c}\hlstd{(}\hlnum{100.0}\hlstd{,} \hlnum{87.1}\hlstd{)}
\hlstd{student02} \hlkwb{<-} \hlkwd{c}\hlstd{(}\hlnum{77.2}\hlstd{,} \hlnum{88.9}\hlstd{)}
\hlstd{student03} \hlkwb{<-} \hlkwd{c}\hlstd{(}\hlnum{66.3}\hlstd{,} \hlnum{87.9}\hlstd{)}

\hlstd{students_combined} \hlkwb{<-} \hlkwd{c}\hlstd{(student01, student02, student03)}
\hlstd{grades} \hlkwb{<-} \hlkwd{matrix}\hlstd{(students_combined,} \hlkwc{byrow} \hlstd{= T,} \hlkwc{nrow} \hlstd{=} \hlnum{3}\hlstd{)}

\hlcom{## Add a new student row with `rbind()`}
\hlstd{student04} \hlkwb{<-} \hlkwd{c}\hlstd{(}\hlnum{95.2}\hlstd{,} \hlnum{94.1}\hlstd{)}
\hlstd{grades} \hlkwb{<-} \hlkwd{rbind}\hlstd{(grades, student04)}

\hlcom{## Add a new assignment column with `cbind()`}
\hlstd{assignment04} \hlkwb{<-} \hlkwd{c}\hlstd{(}\hlnum{92.1}\hlstd{,} \hlnum{84.3}\hlstd{,} \hlnum{75.1}\hlstd{,} \hlnum{97.8}\hlstd{)}
\hlstd{grades} \hlkwb{<-} \hlkwd{cbind}\hlstd{(grades, assignment04)}

\hlcom{## Add the following names to columns and rows using `rownames()` and `colnames()`}
\hlstd{assignments} \hlkwb{<-} \hlkwd{c}\hlstd{(}\hlstr{"Assignment 1"}\hlstd{,} \hlstr{"Assignment 2"}\hlstd{,} \hlstr{"Assignment 3"}\hlstd{)}
\hlstd{students} \hlkwb{<-} \hlkwd{c}\hlstd{(}\hlstr{"Florinda Baird"}\hlstd{,} \hlstr{"Jinny Foss"}\hlstd{,} \hlstr{"Lou Purvis"}\hlstd{,} \hlstr{"Nola Maloney"}\hlstd{)}

\hlkwd{rownames}\hlstd{(grades)} \hlkwb{<-} \hlstd{students}
\hlkwd{colnames}\hlstd{(grades)} \hlkwb{<-} \hlstd{assignments}

\hlcom{## Total points for each assignment using `colSums()`}
\hlkwd{colSums}\hlstd{(grades)}
\end{alltt}
\begin{verbatim}
## Assignment 1 Assignment 2 Assignment 3 
##        338.7        358.0        349.3
\end{verbatim}
\begin{alltt}
\hlcom{## Total points for each student using `rowSums()`}
\hlkwd{rowSums}\hlstd{(grades)}
\end{alltt}
\begin{verbatim}
## Florinda Baird     Jinny Foss     Lou Purvis   Nola Maloney 
##          279.2          250.4          229.3          287.1
\end{verbatim}
\begin{alltt}
\hlcom{## Matrix with 10% and add it to grades}
\hlstd{weighted_grades} \hlkwb{<-} \hlstd{grades} \hlopt{*} \hlnum{0.1} \hlopt{+} \hlstd{grades}

\hlcom{## Create a factor of book genres using the genres_vector}
\hlcom{## Assign the factor vector to factor_genre_vector}
\hlstd{genres_vector} \hlkwb{<-} \hlkwd{c}\hlstd{(}\hlstr{"Fantasy"}\hlstd{,} \hlstr{"Sci-Fi"}\hlstd{,} \hlstr{"Sci-Fi"}\hlstd{,} \hlstr{"Mystery"}\hlstd{,} \hlstr{"Sci-Fi"}\hlstd{,} \hlstr{"Fantasy"}\hlstd{)}
\hlstd{factor_genre_vector} \hlkwb{<-} \hlkwd{as.factor}\hlstd{(genres_vector)}

\hlcom{## Use the `summary()` function to print a summary of `factor_genre_vector`}
\hlkwd{summary}\hlstd{(factor_genre_vector)}
\end{alltt}
\begin{verbatim}
## Fantasy Mystery  Sci-Fi 
##       2       1       3
\end{verbatim}
\begin{alltt}
\hlcom{## Create ordered factor of book recommendations using the recommendations_vector}
\hlcom{## `no` is the lowest and `yes` is the highest}
\hlstd{recommendations_vector} \hlkwb{<-} \hlkwd{c}\hlstd{(}\hlstr{"neutral"}\hlstd{,} \hlstr{"no"}\hlstd{,} \hlstr{"no"}\hlstd{,} \hlstr{"neutral"}\hlstd{,} \hlstr{"yes"}\hlstd{)}
\hlstd{factor_recommendations_vector} \hlkwb{<-} \hlkwd{factor}\hlstd{(}
  \hlstd{recommendations_vector,}
  \hlkwc{ordered} \hlstd{=} \hlnum{TRUE}\hlstd{,}
  \hlkwc{levels} \hlstd{=} \hlkwd{c}\hlstd{(}\hlstr{"no"}\hlstd{,} \hlstr{"neutral"}\hlstd{,} \hlstr{"yes"}\hlstd{)}
\hlstd{)}

\hlcom{## Use the `summary()` function to print a summary of `factor_recommendations_vector`}
\hlkwd{summary}\hlstd{(factor_recommendations_vector)}
\end{alltt}
\begin{verbatim}
##      no neutral     yes 
##       2       2       1
\end{verbatim}
\begin{alltt}
\hlcom{## Using the built-in `mtcars` dataset, view the first few rows using the `head()` function}
\hlkwd{head}\hlstd{(mtcars)}
\end{alltt}
\begin{verbatim}
##                    mpg cyl disp  hp drat    wt  qsec vs am gear carb
## Mazda RX4         21.0   6  160 110 3.90 2.620 16.46  0  1    4    4
## Mazda RX4 Wag     21.0   6  160 110 3.90 2.875 17.02  0  1    4    4
## Datsun 710        22.8   4  108  93 3.85 2.320 18.61  1  1    4    1
## Hornet 4 Drive    21.4   6  258 110 3.08 3.215 19.44  1  0    3    1
## Hornet Sportabout 18.7   8  360 175 3.15 3.440 17.02  0  0    3    2
## Valiant           18.1   6  225 105 2.76 3.460 20.22  1  0    3    1
\end{verbatim}
\begin{alltt}
\hlcom{## Using the built-in mtcars dataset, view the last few rows using the `tail()` function}
\hlkwd{tail}\hlstd{(mtcars)}
\end{alltt}
\begin{verbatim}
##                 mpg cyl  disp  hp drat    wt qsec vs am gear carb
## Porsche 914-2  26.0   4 120.3  91 4.43 2.140 16.7  0  1    5    2
## Lotus Europa   30.4   4  95.1 113 3.77 1.513 16.9  1  1    5    2
## Ford Pantera L 15.8   8 351.0 264 4.22 3.170 14.5  0  1    5    4
## Ferrari Dino   19.7   6 145.0 175 3.62 2.770 15.5  0  1    5    6
## Maserati Bora  15.0   8 301.0 335 3.54 3.570 14.6  0  1    5    8
## Volvo 142E     21.4   4 121.0 109 4.11 2.780 18.6  1  1    4    2
\end{verbatim}
\begin{alltt}
\hlcom{## Create a dataframe called characters_df using the following information from LOTR}
\hlstd{name} \hlkwb{<-} \hlkwd{c}\hlstd{(}\hlstr{"Aragon"}\hlstd{,} \hlstr{"Bilbo"}\hlstd{,} \hlstr{"Frodo"}\hlstd{,} \hlstr{"Galadriel"}\hlstd{,} \hlstr{"Sam"}\hlstd{,} \hlstr{"Gandalf"}\hlstd{,} \hlstr{"Legolas"}\hlstd{,} \hlstr{"Sauron"}\hlstd{,} \hlstr{"Gollum"}\hlstd{)}
\hlstd{race} \hlkwb{<-} \hlkwd{c}\hlstd{(}\hlstr{"Men"}\hlstd{,} \hlstr{"Hobbit"}\hlstd{,} \hlstr{"Hobbit"}\hlstd{,} \hlstr{"Elf"}\hlstd{,} \hlstr{"Hobbit"}\hlstd{,} \hlstr{"Maia"}\hlstd{,} \hlstr{"Elf"}\hlstd{,} \hlstr{"Maia"}\hlstd{,} \hlstr{"Hobbit"}\hlstd{)}
\hlstd{in_fellowship} \hlkwb{<-} \hlkwd{c}\hlstd{(}\hlnum{TRUE}\hlstd{,} \hlnum{FALSE}\hlstd{,} \hlnum{TRUE}\hlstd{,} \hlnum{FALSE}\hlstd{,} \hlnum{TRUE}\hlstd{,} \hlnum{TRUE}\hlstd{,} \hlnum{TRUE}\hlstd{,} \hlnum{FALSE}\hlstd{,} \hlnum{FALSE}\hlstd{)}
\hlstd{ring_bearer} \hlkwb{<-} \hlkwd{c}\hlstd{(}\hlnum{FALSE}\hlstd{,} \hlnum{TRUE}\hlstd{,} \hlnum{TRUE}\hlstd{,} \hlnum{FALSE}\hlstd{,} \hlnum{TRUE}\hlstd{,} \hlnum{TRUE}\hlstd{,} \hlnum{FALSE}\hlstd{,} \hlnum{TRUE}\hlstd{,} \hlnum{TRUE}\hlstd{)}
\hlstd{age} \hlkwb{<-} \hlkwd{c}\hlstd{(}\hlnum{88}\hlstd{,} \hlnum{129}\hlstd{,} \hlnum{51}\hlstd{,} \hlnum{7000}\hlstd{,} \hlnum{36}\hlstd{,} \hlnum{2019}\hlstd{,} \hlnum{2931}\hlstd{,} \hlnum{7052}\hlstd{,} \hlnum{589}\hlstd{)}

\hlstd{characters_df} \hlkwb{<-} \hlkwd{data.frame}\hlstd{(name, race, in_fellowship, ring_bearer, age)}

\hlcom{## Sorting the characters_df by age using the order function and assign the result to the sorted_characters_df}
\hlstd{sorted_characters_df} \hlkwb{<-} \hlstd{characters_df[}\hlkwd{order}\hlstd{(age), ]}
\hlcom{## Use `head()` to output the first few rows of `sorted_characters_df`}
\hlkwd{head}\hlstd{(sorted_characters_df)}
\end{alltt}
\begin{verbatim}
##      name   race in_fellowship ring_bearer  age
## 5     Sam Hobbit          TRUE        TRUE   36
## 3   Frodo Hobbit          TRUE        TRUE   51
## 1  Aragon    Men          TRUE       FALSE   88
## 2   Bilbo Hobbit         FALSE        TRUE  129
## 9  Gollum Hobbit         FALSE        TRUE  589
## 6 Gandalf   Maia          TRUE        TRUE 2019
\end{verbatim}
\begin{alltt}
\hlcom{## Select all of the ring bearers from the dataframe and assign it to ringbearers_df}
\hlstd{ringbearers_df} \hlkwb{<-} \hlstd{characters_df[characters_df}\hlopt{$}\hlstd{ring_bearer} \hlopt{==} \hlnum{TRUE}\hlstd{,]}
\hlcom{## Use `head()` to output the first few rows of `ringbearers_df`}
\hlkwd{head}\hlstd{(ringbearers_df)}
\end{alltt}
\begin{verbatim}
##      name   race in_fellowship ring_bearer  age
## 2   Bilbo Hobbit         FALSE        TRUE  129
## 3   Frodo Hobbit          TRUE        TRUE   51
## 5     Sam Hobbit          TRUE        TRUE   36
## 6 Gandalf   Maia          TRUE        TRUE 2019
## 8  Sauron   Maia         FALSE        TRUE 7052
## 9  Gollum Hobbit         FALSE        TRUE  589
\end{verbatim}
\end{kframe}
\end{knitrout}

The R session information (including the OS info, R version and all
packages used):

\begin{knitrout}
\definecolor{shadecolor}{rgb}{0.969, 0.969, 0.969}\color{fgcolor}\begin{kframe}
\begin{alltt}
\hlkwd{sessionInfo}\hlstd{()}
\end{alltt}
\begin{verbatim}
## R version 4.2.2 (2022-10-31)
## Platform: aarch64-apple-darwin20 (64-bit)
## Running under: macOS Monterey 12.5.1
## 
## Matrix products: default
## LAPACK: /Library/Frameworks/R.framework/Versions/4.2-arm64/Resources/lib/libRlapack.dylib
## 
## locale:
## [1] en_US.UTF-8/en_US.UTF-8/en_US.UTF-8/C/en_US.UTF-8/en_US.UTF-8
## 
## attached base packages:
## [1] stats     graphics  grDevices utils     datasets  methods   base     
## 
## other attached packages:
## [1] knitr_1.41
## 
## loaded via a namespace (and not attached):
##  [1] Rcpp_1.0.9      digest_0.6.30   lifecycle_1.0.3 DBI_1.1.3       magrittr_2.0.3 
##  [6] evaluate_0.18   RSQLite_2.2.19  highr_0.9       stringi_1.7.8   cachem_1.0.6   
## [11] rlang_1.0.6     cli_3.4.1       blob_1.2.3      vctrs_0.5.1     rmarkdown_2.18 
## [16] tools_4.2.2     stringr_1.5.0   bit64_4.0.5     glue_1.6.2      tinytex_0.42   
## [21] bit_4.0.5       xfun_0.35       yaml_2.3.6      fastmap_1.1.0   compiler_4.2.2 
## [26] memoise_2.0.1   htmltools_0.5.4
\end{verbatim}
\begin{alltt}
\hlkwd{Sys.time}\hlstd{()}
\end{alltt}
\begin{verbatim}
## [1] "2022-12-11 09:20:11 MST"
\end{verbatim}
\end{kframe}
\end{knitrout}


\end{document}
